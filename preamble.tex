%%% preamble.tex
%
% This file is for including LaTeX libraries or packages and defining your own
% commands.
%
% NOTE: The glossaries package loaded by tauthesis.cls throws a warning: No
% language module detected for 'finnish'. You can safely ignore this. All
% other warnings should be taken care of, before your thesis is submitted!

%%%%% Your packages.
%
% Before adding packages, see if they can be found in tauthesis.cls already.
% If you're not sure that you need a certain package, don't include it in the
% document! This can dramatically reduce compilation time.

% Graphs
\usepackage{pgfplots}
\pgfplotsset{compat=1.15}

% Subfigures and wrapping text
% \usepackage{subcaption}

%% Theorem environments and their numbering.
%
% Define both English and Finnish theorem types. These all follow the same
% counter. See the documentation of amsthm to see how these can be changed to
% suit your needs, if necessary.
%

\usepackage{amsthm}

\theoremstyle{definition}

\newtheorem{definition}{Definition}[chapter]
\newtheorem{theorem}[definition]{Theorem}
\newtheorem{lemma}[definition]{Lemma}
\newtheorem{corollary}[definition]{Corollary}
\newtheorem{example}[definition]{Example}

\newtheorem{maaritelma}[definition]{Määritelmä}
\newtheorem{lause}[definition]{Lause}
\newtheorem{apulause}[definition]{Apulause}
\newtheorem{seurauslause}[definition]{Seurauslause}
\newtheorem{esimerkki}[definition]{Esimerkki}

% Mathematics packages
\usepackage{mathtools, amssymb}
%\usepackage{bm}

% Chemistry packages
% \usepackage{chemfig}
% \usepackage[version=4]{mhchem}

% Text hyperlinking
% \usepackage{hyperref}
% \hypersetup{hidelinks}

% (SI) unit handling
\usepackage{siunitx}

\sisetup{
    detect-all,
    math-sf=\mathrm,
    exponent-product=\cdot,
    output-decimal-marker={,} % for theses in FINNISH!
}

%%%%% Your commands.

% Print verbatim LaTeX commands
\newcommand{\verbcommand}[1]{\texttt{\textbackslash #1}}

% Command for formatting code.

\newcommand\code[1]{\texttt{#1}}

% A delimiter command for the norm of a vector with mathtools.

\DeclarePairedDelimiter\norm{\lVert}{\rVert}

% Basic theorems in Finnish and in English.
% Remove [chapter] if you wish a simply
% running enumeration.
% \newtheorem{lause}{Lause}[chapter]
% \newtheorem{theorem}[lause]{Theorem}

% \newtheorem{apulause}[lause]{Apulause}
% \newtheorem{lemma}[lause]{Lemma}

% Use these versions for individually
% enumerated lemmas
% \newtheorem{apulause}{Apulause}[chapter]
% \newtheorem{lemma}{Lemma}[chapter]

% Definition style
% \theoremstyle{definition}
% \newtheorem{maaritelma}{Määritelmä}[chapter]
% \newtheorem{definition}[maaritelma]{Definition}
% examples in this style

%%%%% Glossary information.

% Use the following lines ONLY if you need more
% than one glossary. The first argument specifies
% a type label for the glossary and the second
% the displayed name.
% \newglossary*{symbs}{Symbols}
% \newglossary{label}{Displayed name}
% ...

\makeglossaries

% Use this line if using the default glossary.
% Otherwise comment out.

\loadglsentries[main]{tex/sanasto.tex}

% Use this line if using more than one glossary.
% Otherwise comment out.
% \loadglsentries[symbs]{tex/sanasto2.tex}

%%%%% Citation information.

% Commonly used bibliography modifications.
% Feel free to play around with them.

%\ExecuteBibliographyOptions{%
%sorting=none,
%maxbibnames=99,
%maxcitenames=2,
%giveninits=true,
%uniquename=init,
%sortcites,
%sortlocale=fin}

%\DefineBibliographyStrings{finnish}{%
%    in = {},
%    pages = {s.},
%    page = {s.}
%}
%\DefineBibliographyStrings{english}{%
%    in = {},
%    pages = {pp.},
%    page = {p.}
%}
%
%\DeclareNameAlias{sortname}{last-first}
%\DeclareNameAlias{author}{last-first}

%\DeclareFieldFormat[%
%    article,inbook,incollection,inproceedings,
%    patent,thesis,unpublished]{citetitle}{#1\isdot}
%\DeclareFieldFormat[%
%    article,inbook,incollection,inproceedings,
%    patent,thesis,unpublished]{title}{#1\isdot}
%\DeclareFieldFormat{pagetotal}{#1 \bibstring{page}}

%\AtBeginBibliography{\renewcommand*{\makelabel}[1]{#1\hss}}

%\DefineBibliographyExtras{english}{\let\finalandcomma=\empty}

\addbibresource{tex/references.bib}
