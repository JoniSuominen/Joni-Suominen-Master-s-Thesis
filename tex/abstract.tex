The colours of a scene, as recorded by a camera before any processing, differ significantly from those seen by the human eye. The reason is simple: no digital image sensor has the same spectral sensitivities as the human eye. No such sensor can be produced reliably enough due to variations in the manufacturing process, and even if one could, there are often tradeoffs, such as noise performance, which take priority. It is then compensated for in software, inside the processing unit called the Image Signal Processor (ISP), where it is often referred to as colour correction.

This thesis presents a method for using Penalized B-splines (P-splines) for colour correction. B-splines, being piecewise polynomials with minimal support, are a natural expansion of the polynomial methods previously proposed in the literature. In addition, regularization is included in the proposed model through second-order differences of coefficients to impose smoothness in the solution. This allows for more complex models to be fitted without overfitting and robustness to noise.

In addition, a comprehensive comparison is conducted against the state-of-the-art methods, such as Root-Polynomial Regression and Neural Networks.
For this purpose, an open-source test bench for evaluating colour correction algorithms is published. The study shows that the method performs comparably or better than the previous methods across different cameras and training datasets, with minimal tuning required. The disadvantage of the proposed model is that it is not inherently invariant to exposure changes. As such, the effects of penalization with regard to exposure and the nature of the transformation are explored.
