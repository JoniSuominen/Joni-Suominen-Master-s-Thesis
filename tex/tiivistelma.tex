Kameran tallentamat värit, ilman jatkokäsittelyä, eroavat merkitsevästi siitä, miten ihminen näkee ne. Syy tähän on yksinkertainen: yhdelläkään kamerasensorilla ei ole samanlaista spektrivastetta, kuin ihmissilmällä. Tämänkaltaista kuvasensoria olisi mahdotonta valmistaa luotettavasti valmistusprosesseissa esiintyvien vaihtelevuuksien vuoksi, jonka lisäksi kuvasensorin kehitykseen liittyy kompromisseja, kuten kohinan minimointi, jotka ovat etusijalla. Tämän vuoksi värintoistokykyä kompensoidaan kuvasignaaliprosessorilla (ISP) suoritettavalla ohjelmistolla, missä vaihetta kutsutaan yleisimmin värinkorjaukseksi.

Tässä diplomityössä esitellään menetelmä värinkorjaukseen Penalised B-spline (P-spline) -mallia hyödyntäen. B-splinit, jotka ovat matemaattisesti määritelty paloittain jaettuina polynomeina vähäisellä tuella, ovat luonnollinen jatke aiemmin kirjallisuudessa esitellyille polynomiaalisille menetelmille. Lisäksi esitellyssä mallissa hyödynnetään regularisointia kertoimien toisen asteen erotuksien avulla sileämmän funktion saavuttamiseksi. Tämä mahdollistaa monimutkaisempien mallien hyödyntämisen ilman ylisovittamista sekä vähentää kohinan vaikutusta.

Lisäksi työssä esitellään kattava vertailu uusimpiin menetelmiin, kuten Root-Polynomial regressiomalliin ja neuroverkkoihin. Tätä tarkoitusta varten, työn oheistuotteena julkistetaan avoimen lähdekoodin vertailualusta värinkorjausalgoritmeille. Tutkimus osoittaa, että esitelty metodi suoriutuu joko vastaavasti tai paremmin kilpailevia malleja vastaan, kun vertailu suoritetaan eri kameroilla ja koulutusdatalla, minimaalisilla hienosäädöillä. Ehdotetun mallin huonoksi puoleksi todetaan, että sillä ei ole luontaisesti invariantti valotuksen muutoksille. Tämän vuoksi tutkimuksessa syvennytään siihen, miten regularisaatio vaikuttaa mallien suorituskykyyn eri valotustasoilla. Tämän lisäksi syvennytään regressiomallin muunnoksen luonteeseen.